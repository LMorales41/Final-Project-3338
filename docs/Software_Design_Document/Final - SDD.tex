\documentclass{article}
\usepackage{geometry}
\usepackage{graphicx}
\usepackage{hyperref}
\geometry{margin=1in}
\graphicspath{  {C:/Users/ccuahue/OneDrive - Cal State LA/Pictures/Screenshots}  }

%Cover Page
\title{Software Design Document (SDD) \\Metro Bike Share Project}
\author{Smart City}
\date{May 6, 2025}

\begin{document}

%Cover Page
\maketitle
\newpage

%Table of Contents 
\tableofcontents
\newpage

%Version Table
\begin{center}

\section*{Version Table}
\begin{tabular}{|c|c|c|c|}
\hline
\textbf{Version} & \textbf{Date} & \textbf{Description} \\
\hline
Snapshot 1 & 4-24-2025 & Include all requirements\\
\hline
Snapshot 2 & 4-30-2025 & Revise, include more data, add a glossary\\
\hline
Snapshot 3 & 5-2-2025 & Add images to User Interface and System Architecture Sections\\
\hline
Final & 5-6-2025 & Add a reference \\
\hline
\end{tabular}
\end{center}

\newpage

%Introduction Section
\section{Introduction}
\subsection{Purpose}
This document is to explain in detail the functions that the application will perform. The document will inform readers as to what the application will do. The purpose of this product is to visualize pedestrian and bicycle data to find and identify problem areas and safest navigation routes.

\subsection{Intended Audience}
The main audience of the software requirements specifications document are developers, project managers, and testers. The SRS contains information about each project such as what the project is, what each of its UI elements should do, and what dependencies each project may have. It is suggested that you first look at the table of contents for any topics you may be looking for, if not then quickly skim the document to get a better understanding of the projects. If you are a developer or project manager it is suggested that you look into section 4 of the SRS so that you may check if project requirements are being met. If you are a tester it is suggested that you look into section 3 so that you have a better understanding of how the user interfaces should work.

\subsection{Overview}
\begin{itemize}
	\item \textbf{Metro Bike Share Real Time(Web \& Android App)}\\
The focus of this project is to aid in the realization of Vision 0 is the greater Los Angeles area. This is done by modeling real time Metro Bike Share stations within the city, in conjunction with bike accidents around the city. Currently the idea is to allow the user to draw the “safest” path by avoiding areas of the city where major accidents have occured.

	\item \textbf{Metro Bike Share Historical Data Visualization (Web * Android App) }\\
The purpose of this project is to help visualize the Metro Bike Share data. The stations are shown as feature layers on the map. Clicking the station displays information such as number of trips and busiest day. Station icons vary in size and color depending on what information we want to show.
\end{itemize}

%System Architecture section 
\section{System Architecture}
%add the images 
\includegraphics[width=170mm,scale=0.5]{SA1}
\includegraphics[width=170mm,scale=0.5]{SA2}

\newpage

%User Interface section
\section{User Interface}
\subsection{Overview of User Interface}
\begin{itemize}
	\item Zoom in and out to change visibility
	\item Hover over markers displays an info window about the station
	\item User location button to prompt user for their location
	\item Reset map to its original state using reset button
	\item Toggle heat map on \& off
	\item Draw a Polyline to allow user to draw their path by clicking anywhere on the map. This can toggle on/off
	\item Click on station to be highlighted \& its corresponding marker will be animated
	\item Filter by city with the drop down menu
	\item Click on a station marker to display directions \& a polyline of how to get from the user’s location to the clicked station
	\item Use google map’s default features
\end{itemize}
\subsection{Screen Frameworks or Images}
\includegraphics[width=170mm,scale=0.5]{UI1}
\includegraphics[width=170mm,scale=0.5]{UI2}
\includegraphics[width=170mm,scale=0.5]{UI3}
\includegraphics[width=170mm,scale=0.5]{UI4}
\includegraphics[height = 110mm,width=145mm,scale=0.5]{UI5}\\
\includegraphics[height = 110mm,width=145mm,scale=0.5]{UI6}

\subsection{User Interface Flow Model}
\includegraphics[width=170mm,scale=0.5]{UI7}\\

%Glossary section 
\section{Glossary}
\begin{tabular}{|c|c|}
\hline
SRS & Specific Requirements Specifications\\
\hline
SDD & Software Design Document\\
\hline
\end{tabular}

%Reference section
\section{References}
\begin{tabular}{|c|p{10cm}|}
\hline
Google Maps API & The Maps JavaScript API lets you customize maps with your own content and imagery for display on web pages and mobile devices. The Maps JavaScript API features four basic map types (roadmap, satellite, hybrid, and terrain) which you can modify using layers and styles, controls and events, and various services and libraries. \href{https://developers.google.com/maps/documentation/javascript/tutorial}{link}\\
\hline
Metro Bike Data & Anonymized Metro Bike Share trip data for data collection \href{https://bikeshare.metro.net/about/data/} {link}\\
\hline
Google's Direction API & Used to retrieve a JSON object containing directions information between points \href{https://developers.google.com/maps/documentation/directions/start} {link}\\
\hline
Maps SDK for Android & Adds functionality to elements within the map \href{https://developers.google.com/maps/documentation/android-sdk/intro} {link}\\
\hline
GeoHub & Data collection for the city of Los Angeles \href{http://geohub.lacity.org/} {link}\\
\hline
Android Studio & Used to develop MBSRT android version \href{https://developer.android.com/docs} {link}\\
\hline
Firebase & Used to authenticate and store user data. \href{https://firebase.google.com/} {link}\\
\hline
Jupyter Notebook & Organize and manipulate data. \href{https://jupyter.org/} {link}\\
\hline

\end{tabular}
\end{document}